{ \large \bfseries 1.Σκοπός της Άσκησης}\\ % title 1

\begin{justify}

Ο σκοπός της Άσκησης αυτής είναι η σχεδίαση ενός πλήρους λειτουργικού επεξεργαστή μονο κύκλου με την χρήση της γλώσσας \textlatin{VHDL}. 
Η σχεδίαση χωρίστηκε σε 3 φάσεις με σκοπό την πλήρης κατανόηση της λειτουργίας του επεξεργαστή.\\\\
Στην 1\textsuperscript{η} φάση σχεδιάστηκε μια μονάδα αριθμιτικών και λογικών πράξεων
(\textlatin{ALU}) καθώς και ένα αρχείο καταχωρητών(\textlatin{Register File}).\\\\
Στην 2\textsuperscript{η} σχεδιάστηκαν οι βασικές βαθμίδες του \textlatin{Datapath}
του επεξαργαστή. Αυτές είναι βαθμίδα ανάκλησης εντολών\textlatin{(IFSTAGE)}, αποκωδηκοποίησης εντολών\textlatin{(DECSTAGE)}, εκτέλεσης εντολών\textlatin{(EXSTAGE)} και πρόσβασης μνήμης \textlatin{(MEMSTAGE)}. Για την σχεδίαση χρειάστηκαν η \textlatin{(ALU)} και ο \textlatin{(Register file)} από την προηγούμενη φάση καθώς και με την χρήση του εργαλείου \textlatin{Xillinx Core Generator} υλοποιήθηκαν 2 μνήμες, μια \textlatin{ROM 1024x32} για την αποθήκευση των βασικών εντολών και μια κύρια μνήμη \textlatin{RAM 1024x32}.\\\\
Τέλος, στην 3\textsuperscript{η} φάση συνδέθηκαν τα επιμέρους στοιχεία για να κατασκευαστεί το ενιαίο \textlatin{DATAPATH} καθώς και υλοποιήθηκε το \textlatin{CONTROL} του επεξεραστή το οποίο παράγει τα σωστά σήματα ελέγχου για την κάθε εντολή.
\end{justify}


\vspace{0.5cm}
